%%%%%%%%%%%%%%%%%%%%%%%%%%%%%%%%%%%%%%%%%
% Thin Formal Letter
% LaTeX Template
% Version 1.11 (8/12/12)
% Compile with XeLaTeX

%----------------------------------------------------------------------------------------
%	DOCUMENT CONFIGURATIONS
%----------------------------------------------------------------------------------------

\documentclass[a4paper]{letter}

\usepackage[margin=2.5cm]{geometry}  %set margin width all around page
	
\addtolength{\voffset}{0in} % To push the letter further up on the page

\longindentation=0pt % Un-commenting this line will push the closing "Sincerely," to the left of the page

%----------------------------------------------------------------------------------------
%	 NAME & ADDRESS
%----------------------------------------------------------------------------------------

\signature{John L. Godlee} % Your name for the signature at the bottom

\address{Crew Building \\ Alexander Crum Brown Road \\ The King's Buildings \\ Edinburgh \\ EH9 3FF \\ United Kingdom \\ \\ johngodlee@gmail.com} % Your address and phone number

%----------------------------------------------------------------------------------------

\begin{document}

% It should be concise and explain why the content of the paper is significant, placing the findings in the context of existing work and why it fits the scope of the journal. Confirm that neither the manuscript nor any parts of its content are currently under consideration or published in another journal. Any prior submissions of the manuscript to MDPI journals must be acknowledged. The names of proposed and excluded reviewers should be provided in the submission system, not in the cover letter.


%----------------------------------------------------------------------------------------
%	ADDRESSEE 
%----------------------------------------------------------------------------------------

\begin{letter}{} % Name/title of the addressee

%----------------------------------------------------------------------------------------
%	LETTER CONTENT 
%----------------------------------------------------------------------------------------

	\opening{\textbf{Dear Dr. Lu\'{i}s Catarino, Prof. Maria Romeiras,}}
 
I enclose a manuscript for submission to Diversity journal's special issue: `Biodiversity of Vegetation and Flora in Tropical Africa'. Our manuscript is entitled `Diversity and structure of an arid miombo woodland at the western edge of the miombo ecoregion, southwest Angola'. The article comprises a plot-based study of the species composition and physical structure of woodlands in Bicuar National Park, southwest Angola. We compared plots in Bicuar National Park with similar plots in southern Tanzania, Gorongosa National Park in Mozambique, and southern Democratic Republic of Congo to set our results in the context of the wider miombo ecoregion. The woodlands of Bicuar National Park within the larger Hu\'{i}la plateau area represent the extreme western extent of the miombo ecoregion and are markedly drier than woodlands found elsewhere. They are also historically understudied compared to woodlands found further east in southern Africa despite being identified as a region of high African endemism by previous checklist type studies, mainly due to the long running civil war, which made research of this sort in this region extremely difficult until around 2012. 

In our study we found that the woodlands of Bicuar National Park represent a unique low biomass miombo woodland type with few large trees, with 27 species not found in other plots within our study. In a comparison of plots in areas of the National Park which had been previously degraded by shifting cultivation farming, we found that these previously farmed plots contained a greater diversity of tree species than non-degraded plots. Our study lends support that Bicuar National Park and the greater Hu\'{i}la plateau should be considered a conservation priority due to their unique woodland composition. This study contributes to the growing appreciation of the regional heterogeneity of miombo woodlands and their importance for biodiversity.

We think this study fits well with the general themes of this special issue of Diversity. Our article draws attention to, and presents comprehensive results of the structure and composition of, a traditionally understudied part of the miombo woodland ecoregion. We hope this article shed light on southwest Angola as a centre of endemism that is currently under threat from deforestation under a developing rural economy.

We confirm that the data for plots in Bicuar National Park have not been published elsewhere. The data for plots in Tanzania, Mozambique and the Democratic Republic of Congo have all been previously published in isolation. The current manuscript uses this data as a point of comparison for Bicuar National Park. Each of the authors confirms that neither the manuscript nor any parts of its content are currently under consideration or published in another journal. All authors additionally have approved the contents of this article and have agreed to the journal's submission policies. 

We thank you for your consideration.

\vspace{\parskip} % Extra whitespace for aesthetics
\closing{Yours,}
\vspace{2\parskip} % Extra whitespace for aesthetics

% \ps{P.S. You can find additional information attached to this letter.} % Postscript text, comment this line to remove it

% \encl{Copyright permission form} % Enclosures with the letter, comment this line to remove it

%----------------------------------------------------------------------------------------

\end{letter}
 
\end{document}

